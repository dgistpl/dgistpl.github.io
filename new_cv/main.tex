%-------------------------
% Resume in Latex
% Author : Jake Gutierrez
% Based off of: https://github.com/sb2nov/resume
% License : MIT
%------------------------

\documentclass[letterpaper,11pt]{article}

\usepackage{latexsym}
\usepackage{kotex}
\usepackage[empty]{fullpage}
\usepackage{titlesec}
\usepackage{marvosym}
\usepackage[usenames,dvipsnames]{color}
\usepackage{verbatim}
\usepackage{enumitem}
\usepackage[hidelinks]{hyperref}
\usepackage{fancyhdr}
\usepackage[english]{babel}
\usepackage{tabularx}
\usepackage{fontawesome5}
\usepackage{multicol}
\setlength{\multicolsep}{-3.0pt}
\setlength{\columnsep}{-1pt}
\input{glyphtounicode}
\usepackage{multirow}


%----------FONT OPTIONS----------
% sans-serif
% \usepackage[sfdefault]{FiraSans}
% \usepackage[sfdefault]{roboto}
% \usepackage[sfdefault]{noto-sans}
% \usepackage[default]{sourcesanspro}

% serif
% \usepackage{CormorantGaramond}
% \usepackage{charter}


\pagestyle{fancy}
\fancyhf{} % clear all header and footer fields
\fancyfoot{}
\renewcommand{\headrulewidth}{0pt}
\renewcommand{\footrulewidth}{0pt}

% Adjust margins
\addtolength{\oddsidemargin}{-0.6in}
\addtolength{\evensidemargin}{-0.5in}
\addtolength{\textwidth}{1.19in}
\addtolength{\topmargin}{-.7in}
\addtolength{\textheight}{1.4in}

\urlstyle{same}

\raggedbottom
\raggedright
\setlength{\tabcolsep}{0in}

% Sections formatting
\titleformat{\section}{
  \vspace{-4pt}\scshape\raggedright\large\bfseries
}{}{0em}{}[\color{black}\titlerule \vspace{-5pt}]

% Ensure that generate pdf is machine readable/ATS parsable
\pdfgentounicode=1

%-------------------------
% Custom commands
\newcommand{\resumeItem}[1]{
  \item\small{
    {#1 \vspace{-2pt}}
  }
}

\newcommand{\classesList}[4]{
    \item\small{
        {#1 #2 #3 #4 \vspace{-2pt}}
  }
}

\newcommand{\resumeSubheadingTwo}[2]{
  \vspace{-2pt}\item
    \begin{tabular*}{1.0\textwidth}[t]{l@{\extracolsep{\fill}}r}
      %\textbf{#1} & \textbf{\small #2} \\
      \textit{#1} & \textit{#2} \\
    \end{tabular*}\vspace{-7pt}
}
\newcommand{\resumeSubheading}[4]{
  \vspace{-2pt}\item
    \begin{tabular*}{1.0\textwidth}[t]{l@{\extracolsep{\fill}}r}
      \textbf{#1} & \textbf{\small #2} \\
      \textit{\small#3} & \textit{\small #4} \\
    \end{tabular*}\vspace{-7pt}
}
\newcommand{\resumeSubSubheading}[2]{
    \item
    \begin{tabular*}{0.97\textwidth}{l@{\extracolsep{\fill}}r}
      \textit{\small#1} & \textit{\small #2} \\
    \end{tabular*}\vspace{-7pt}
}

\newcommand{\resumeProjectHeading}[2]{
    \item
    \begin{tabular*}{1.001\textwidth}{l@{\extracolsep{\fill}}r}
      \small#1 & \textbf{\small #2}\\
    \end{tabular*}\vspace{-7pt}
}

\newcommand{\resumeSubItem}[1]{\resumeItem{#1}\vspace{-4pt}}

\renewcommand\labelitemi{$\vcenter{\hbox{\tiny$\bullet$}}$}
\renewcommand\labelitemii{$\vcenter{\hbox{\tiny$\bullet$}}$}

\newcommand{\resumeSubHeadingListStart}{\begin{itemize}[leftmargin=0.0in, label={}]}
\newcommand{\resumeSubHeadingListEnd}{\end{itemize}}
\newcommand{\resumeItemListStart}{\begin{itemize}}
\newcommand{\resumeItemListEnd}{\end{itemize}\vspace{-5pt}}

%-------------------------------------------
%%%%%%  RESUME STARTS HERE  %%%%%%%%%%%%%%%%%%%%%%%%%%%%


\begin{document}

%----------HEADING----------
% \begin{tabular*}{\textwidth}{l@{\extracolsep{\fill}}r}
%   \textbf{\href{http://sourabhbajaj.com/}{\Large Sourabh Bajaj}} & Email : \href{mailto:sourabh@sourabhbajaj.com}{sourabh@sourabhbajaj.com}\\
%   \href{http://sourabhbajaj.com/}{http://www.sourabhbajaj.com} & Mobile : +1-123-456-7890 \\
% \end{tabular*}

\begin{center}
    {\Huge \scshape Minseok Jeon} \\ \vspace{1pt}
    Research Professor\\\vspace{1pt}
    %Department of Computer Science and Engineering\\\vspace{1pt}
    Korea University\\\vspace{1pt}
        
    %123 Street Name, Town, State 12345 \\ \vspace{1pt}
    \small \raisebox{-0.1\height}\faPhone\ +82-10-4139-4729 ~ \href{mailto:minseok_jeon@korea.ac.kr}{\raisebox{-0.2\height}\faEnvelope\  \underline{minseok$\_$jeon@korea.ac.kr}} ~ 
    \href{https://minseokjgit.github.io/}{\raisebox{-0.2\height}\faHome\ \underline{https://minseokjgit.github.io/}} 
    \vspace{-8pt}
\end{center}

\section{Research Interests}
I am broadly interested in developing programming language techniques for addressing challenges in various fields, including software engineering and machine learning.
%
Specifically, I take pleasure in designing domain-specific programming languages (DSLs) and developing program synthesis algorithms to address the challenges.
%
In particular, my focus is on developing DSLs and synthesis algorithms for effective program analysis and explainable graph machine learning.


% In particular, my focus is on designing DSLs and synthesis algorithms for effective pointer analysis, a key component in compiler optimization. 
% %
% I am also interested in developing DSLs tailored to identify effective test cases in system software testing.







% I am interested in static program analysis, with a focus on pointer analysis, 
% which is a key component in compiler optimization and various other software engineering techniques.
% I am also interested in software testing to find bugs in system software. 

% \begin{itemize}[itemsep=-1pt, parsep=3pt]
%   \item\small \textbf{Static Program Analysis}, focusing on pointer analysis, for compiler optimizations and automatic detection of software bugs and vulnerabilities.

%   \item\small \textbf{Software testing} for automatically generating effective test inputs to detect bugs in system software.

%   \item\small \textbf{Program Synthesis} for automatically generating programs from domain-specific programming languages.
% \end{itemize}

% To address the research problems, I have designed domain-specific programming languages (DSLs) tailored to the problems and developed program synthesis algorithms that automatically generate programs (solutions) in the DSLs.


%-----------EDUCATION-----------
\section{Education Background}
  \resumeSubHeadingListStart
  \resumeSubheadingTwo
      %{State University}{Mar. 2017 -- Feb 2023}
      {Integrated M.S. $\&$ Ph.D. in Computer Science and Engineering. Korea University}{Mar. 2017 -- Feb 2023}\vspace{-12pt}
      \resumeSubheadingTwo
      %{State University}{Mar. 2017 -- Feb 2023}
      {B.S. in Computer Science and Engineering. Korea University}{Mar. 2011 -- Feb 2017}      
  \resumeSubHeadingListEnd

%------Employment-------

\section{Employment History}
\resumeSubHeadingListStart
    \resumeSubheadingTwo
    {Research Professor. Korea University}{July. 2024 -- Present}\vspace{-10pt}  
    \resumeSubheadingTwo
    {Postdoctoral Researcher. Korea University}{Mar. 2023 -- June. 2024}\vspace{-15pt}  
    % \resumeSubheading
    % {Korea University}{Mar. 2023 -- Aug 2023}
    % {Postdoctoral Researcher. Korea University}{Mar. 2023 -- Present}\vspace{-1pt}      
\resumeSubHeadingListEnd

%------Publications-------
\section{Publications}
Published papers on programming languages in premier conferences
(PLDI 2024, POPL 2022, OOPSLA 2020, OOPSLA 2018, and OOPSLA 2017) and
journal (TOPLAS 2019). 
\begin{enumerate}
  % \item {
  %   \underline{Minseok Jeon}, Jihyeok Park, and Hakjoo Oh.\\
  %   \textit{PL4XGL: A Programming Language Approach to Explainable Graph Learning.}\\ November 2023 (Submitted)
  % }
  \item {
    \underline{Minseok Jeon}, Jihyeok Park, and Hakjoo Oh.\\
    \textit{PL4XGL: A Programming Language Approach to Explainable Graph Learning.}\\
    PLDI 2024 : ACM SIGPLAN Conference on Programming Language Design and Implementation. June 2024
  }  
  \item {
    Jinkook Kim, \underline{Minseok Jeon}, Sejeong Jang, and Hakjoo Oh.\\
    \textit{Automating Endurance Test for Flash-based Storage Devices in Samsung Electronics.}\\
    ICST 2023: IEEE International Conference on Software Testing, Verification and Validation (Industry Track). April 2023
  }
  \item{
    \underline{Minseok Jeon} and Hakjoo Oh.\\
    \textit{Return of CFA: Call-Site Sensitivity Can Be Superior to Object Sensitivity Even for
    Object-Oriented Programs.}\\
    POPL 2022: The 49th ACM SIGPLAN Symposium on Principles of Programming Languages. January 2022
  }
  \item{
    Donghoon Jeon, \underline{Minseok Jeon}, and Hakjoo Oh.\\
    \textit{A Practical Algorithm for Learning Disjunctive Abstraction Heuristics in Static Program
    Analysis.}\\
    IST: Information and Software Technology. July 2021
  }
  \item{
    \underline{Minseok Jeon}, Myungho Lee, and Hakjoo Oh.\\
    \textit{Learning Graph-based Heuristics for Pointer Analysis without Handcrafting ApplicationSpecific Features.}\\
    OOPSLA 2020: ACM Conference on Object-Oriented Programming, Systems, Languages, and Applications. November 2020
  }
  \item{\underline{Minseok Jeon}$^*$, Sehun Jeong$^*$, Sungdeok Cha, and Hakjoo Oh (*co-first authors).\\
  \textit{A Machine-Learning Algorithm with Disjunctive Model for Data-Driven Program Analysis.}\\
  TOPLAS: ACM Transactions on Programming Languages and Systems. June 2019
  }
  \item{\underline{Minseok Jeon}, Sehun Jeong, and Hakjoo Oh.\\
  \textit{Precise and Scalable Points-to Analysis via Data-Driven Context Tunneling.}\\
  OOPSLA 2018: ACM Conference on Object-Oriented Programming, Systems, Languages, and Applications. November 2018
  }
  \item {Sehun Jeong$^*$, \underline{Minseok Jeon}$^*$, Sungdeok Cha, and Hakjoo Oh  (*co-first
  authors).\\\textit{Data-Driven Context-Sensitivity for Points-to Analysis.}\\
  OOPSLA 2017: ACM Conference on Object-Oriented Programming, Systems, Languages, and Applications. October 2017
  }
\end{enumerate}


% %------Ongoing Projects-------
% \section{Ongoing Projects}
% Ongoing research projects with students.
% \begin{enumerate}
%   \item {%Donguk Kim and Minseok Jeon\\
%     \textit{Project-Aware Fault Localization via Synthesizing Suspiciousness Score Updating Rules}\\
%     with Donguk Kim (undergraduate student)
%     }
%   \item {
%     \textit{Programming Language-based Automated Feature Engineering for Graph Neural Networks}\\
%     with Seunghyun Park (undergraduate student)
%     }
%   \item {%Seokhyun Lee and Minseok Jeon\\
%     \textit{Automatically Classifying Minor Revisions in Programming Assignments}\\
%     with Seokhyun Lee (Ph.D. student)
%     }  
% \end{enumerate}
  
% My ongoing projects.
% \begin{enumerate}
%   \item {\textit{Learning Tunneling Hueristics for JavaScript Static Analysis}}
%   \item {\textit{Learning Cominations of Selective Context Sensitivity and Context Tunneling for Java Pointer Analysis}}
%   \item {\textit{Understanding Context Tunneling in Java Pointer Analysis}}
% \end{enumerate}


%------Service-------
\section{Service}
Program committee (PC) members:
\begin{enumerate}
\item{OOPSLA 2024: ACM Conference on Object-Oriented Programming, Systems, Languages, and Applications}
\end{enumerate}


\section{Talks}

\begin{enumerate}
  \item{PL4XGL: A Programming Language Approach to Explainable Graph Learning. Paper presentation at PLDI 2024. Copenhagen, Denmark. June 27 2024.}
  
  \item{PL4XGL: 프로그래밍 언어 기법을 활용한 설명 가능한 그래프 기계학습 방법. KAIST (ProSysLab Seminar). May 03 2024.}
  
  \item{그래프 패턴 언어를 활용하여 다양한 분야의 핵심 문제 접근하기. STAAR Workshop. KAIST. Jan 30 2024}

  \item{Data-Driven Static Analysis. POSTECH. Pohang, Korea. Nov 15 2023.}

  \item{Return of CFA: Call-Site Sensitivity Can Be Superior to Object Sensitivity Even for Object- Oriented Programs. STAAR Workshop. Jeju. Feb 11 2022.}

  \item{Return of CFA: Call-Site Sensitivity Can Be Superior to Object Sensitivity Even for Object- Oriented Programs. Paper presentation at POPL 2022. Philadelphia,  USA. Jan 19 2022.}
 
  \item{Learning Graph-based Heuristics for Pointer Analysis without Handcrafting Application- Specific Features. KSC2020.}
 
  \item{Learning Graph-based Heuristics for Pointer Analysis without Handcrafting Application- Specific Features. Paper presentation at OOPSLA 2020. Online. NOV 20 2020.}
 
  \item{Precise and Scalable Points-to Analysis via Data-Driven Context Tunneling. Paper presen- tation at OOPSLA 2018. BOSTON, USA. NOV 8 2018.}
 
  \item{Data-Driven Context-Sensitivity for Points-to Analysis, KCC 2018. JeJu, Korea.}
 
  \item{Data-Driven Context-Sensitivity for Points-to Analysis, KCSE 2018. Pyeongchang, Korea.}
\end{enumerate}

% Please add the following required packages to your document preamble:
% \usepackage{multirow}

\section{Grants}
\begin{enumerate}
  \item{설명 가능한 그래프 기계학습 방법 개발을 위한 프로그래밍 언어 기술 연구 (Programming Language Technology for Explainable Graph Machine Learning)
  \begin{itemize}
    \item {역할: 책임연구자 (Principal Investigator)}
    \item {지원기관: 한국연구재단}
    \item {기간: 2024/05/01 - 2029/04/30}
    \item {연간 연구비: 연간 직접비 1억원 (총 5억원), 간접비 별도}
  \end{itemize}
}
\end{enumerate}

\section{Teaching Experience}
\begin{itemize}
  \item{{\bf 2024 Fall:} Data Structure (Korea University COSE214)}
\end{itemize}


% \section{Courses}
% \begin{itemize}
%   \item{COSE 214}
% \end{itemize}

\end{document}

