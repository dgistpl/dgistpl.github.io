\documentclass[11pt]{article}

\usepackage{hyperref}
\usepackage{url}
\usepackage{color}
\usepackage{breakurl}
\newcommand{\bburl}[1]{\textcolor{blue}{\url{#1}}}

\newcommand{\Rescue}{\textsc{Rescue}}

 \addtolength{\textwidth}{5cm}
 \addtolength{\textheight}{5cm}
 \addtolength{\hoffset}{-2.5cm}
  \addtolength{\voffset}{-3cm}
 \addtolength{\marginparwidth}{-2cm}

 \newcommand{\myparagraph}[1]{\medskip\noindent{\it \textbf{#1.}}}
\begin{document}
\title{Teaching Statement}
\author{Minseok Jeon}
\date{}

\maketitle



As an educator, my primary goal in teaching is to motivate students to use their acquired knowledge to solve their problems. Teaching is the most important academic activity that has a great impact not only on students but also on instructors, teaching assistants, and academia. In a course, students face existing knowledge, and each of their understandings is graded by exams or assignments. However, I believe that the goal of teaching is not just to transmit existing knowledge but to enable the next generation to use the acquired knowledge or create new knowledge to solve their problems. I have experienced that well-designed course materials play a pivotal role in achieving this goal.








\section{Teaching Experience}
I was a teaching assistant (TA) in a programming language course and designed a system that allowed students to improve it with their ideas. In the course, the students' grades were mainly determined by their programming assignments.
Students submitted their codes, and the submissions are graded by scoring test cases.
Because of programming assignments' characteristics, however, small mistakes (occurring a compile error) often resulted in a poor score (e.g., 0). Students who made such mistakes were often disappointed and requested opportunities to rescue their programs by fixing the minor errors. As I sympathized with the students and wanted to give them a chance, I designed a system named \Rescue~that automatically classifies whether a given code change is a minor revision that corrects only small mistakes. The student resubmitted their revised code, and I accepted the resubmitted code if our system classified the given code change as a minor revision.



A few weeks later, amazing things happened. The students started proposing creative ideas to improve the system and discussing vulnerabilities. I would like to note that the system was an improvised one; improving or analyzing the system was not the main content of the course. The students, however, enjoyed using their talent or knowledge to enhance or analyze the system. For example, some students leveraged their programming language skills, which they had learned in the course, to improve the system to capture minor revisions that corrected typos, which had not been possible before the idea.
Other students used their knowledge of security to analyze the system's vulnerabilities and propose valuable ideas to remove them. I was excited by the proposed ideas and actively discussed and applied them to the system.
I believe that such activities are the essence of education, enabling students to leverage their knowledge and talents to address their problems.
The story of the system now becomes a research paper consisting of various interesting ideas proposed by the students.
This experience has led me to believe that providing interesting content that motivates the student is the most important part of teaching.

% If a content deeply motivates students, they will not just learn existing knowledge but leverage all their knowledge to give innovative ideas to address the problems.

%related to not only course but the students 


% The experience makes me believe that providing interesting unseen contents is the most import part of teaching. If the content enables the students to be deeply motivated by the contents, the course can provide an ideal education. 
% Students will not just learn existing knowledge to complete their assignments. They will try to leverage all their knowledge to give innovative idea for the unseen problem.







\section{Teaching Interests}
My research experience covers a wide range of topics in programming languages and software engineering.
I have conducted various research projects related to static program analysis; I want to teach undergraduate and graduate courses related to my research topic. Then I can provide interesting content related to my research topic for students.
Specifically, I want to teach {\it Programming Languages}, {\it Software Engineering}, {\it Compiler}, and {\it Theory of Computation} courses:







\myparagraph{Programming Language} In this course, I will teach the essential concepts of programming languages. Students will first learn the basic concepts of programming languages, such as syntax and semantics. Then, students will learn advanced concepts such as program analysis or type systems. Students will implement an interpreter and type system of small programming languages to learn the concepts. Additionally, I will provide content to motivate the students to design and implement their own programming languages to solve their problems.





\myparagraph{Software Engineering}
This course aims to teach the process of developing software.
Students will understand software requirements, software design, software development, software analysis, and software maintenance.
I will introduce various recent research projects related to software engineering. Students will use the software engineering techniques, and I will give them a chance to improve them.







\myparagraph{Compiler}
This course aims to teach the key concepts and components of compilers.
Students will learn how a high-level programming language is safely translated into a low-level programming language.
Topics include lexical, syntax, semantic analysis, and optimization.
Students will implement a compiler, including a lexer, parser, translator, and optimizer to learn the concepts. I will also introduce how static analysis can be used to optimize the compiler.






\myparagraph{Theory of Computation}
The goal of this course is to understand the Turing machine and computability.
Students will first learn finite automata, regular expression, and context-free grammar; students will understand their expressiveness and limitations. 
Then, students will learn Turing machine, which is a foundation of computer science. Finally, students will learn the undecidability. I will introduce how the undecidability is related to the static program analysis. 











\section{Advising}
I believe that each student has a unique research topic, and the most important challenge in advising is exploring the talents and interests of the students.
During my Ph.D. course, I informally guided two students, and they successfully published their projects as research papers and obtained their master degrees. Currently, I am also advising two undergraduate students.
When I was advising them, I observed that each student has different talents and interests.
For example, a student is interested in C++ language and has talent in implementation. Another student is interested in theoretical analysis.
Advising should find and maximize such talent and interest. Then, it will lead to outstanding results.
To do so, I will have regular meeting with the students for establishing effective communication and discussion. 
I will communicate with each student at least once a week to guide and discuss their research.
I will also establish regular seminars to enable the students to present their ideas and discuss with other students.
Students will get feedback from other students and me, and they will learn how to present their ideas and give valuable feedback to others.








\end{document}



%\noindent $2000$: I ran weekly review sessions for the FSI
%students (see above) taking second semester calculus. \\
