%\documentclass[a4paper]{article}
\documentclass[a4paper, 11pt]{article}
% Seems like it does not support 9pt and less. Anyways I should stick to 10pt.
%\documentclass[a4paper, 9pt]{article}
\topmargin-2.0cm

\usepackage{fancyhdr}
\usepackage{pagecounting}
\usepackage[dvips]{color}
% NEW COMMAND
% marginsize{left}{right}{top}{bottom}:
%\marginsize{3cm}{2cm}{1cm}{1cm}
%\marginsize{0.85in}{0.85in}{0.625in}{0.625in}
\advance\oddsidemargin-0.65in
%\advance\evensidemargin-1.5cm
\textheight9.2in
\textwidth6.75in
\newcommand\bb[1]{\mbox{\em #1}}
\def\baselinestretch{1.25}
%\pagestyle{empty}
\newcommand{\hsp}{\hspace*{\parindent}}
\definecolor{gray}{rgb}{0.4,0.4,0.4}
\newcommand{\Rescue}{\textsc{Rescue}}

\begin{document}
\pagestyle{fancy}
% Leave Left and Right Header empty.
%\lhead{}
%\rhead{}
%\rhead{\thepage}
%\lhead{\textcolor{gray}{\it Sundar Iyer}}
%\rhead{\textcolor{gray}{\thepage/\totalpages{}}}
\renewcommand{\headrulewidth}{0pt}
\renewcommand{\footrulewidth}{0pt}
%\fancyfoot[C]{\footnotesize \textcolor{gray}{http://www.stanford.edu/$\sim$sundaes/application}}

%\pagestyle{myheadings}
%\markboth{Sundar Iyer}{Sundar Iyer}

%\pagestyle{myheadings}
%\markboth{Sundar Iyer}{Sundar Iyer}

% tHIS kind of makes 10pt to 9 pt.
\begin{small}


\newcommand{\myparagraph}[1]{\medskip\noindent{\it \textbf{#1.}}}

\begin{center}
{\LARGE \bf TEACHING STATEMENT}\\
\vspace*{0.1cm}
{\normalsize Minseok Jeon}
\end{center}
%\vspace*{0.2cm}



As an educator, my primary goal of teaching is to motivate students to use their acquired knowledge to solve their problems. Teaching is the most important academic activity that has a great impact not only on students but also on instructors, teaching assistants, and academia. In a course, students encounter existing knowledge and try to understand it, and each of their understanding is just graded by exams or assignments. However, I believe that the goal of teaching is not just transmitting existing knowledge, but enabling the next generation to use the acquired knowledge or create new knowledge to solve problems. I have experienced that well-designed course materials play a pivotal role in achieving this goal








\myparagraph{Teaching Experience}
I was a teaching assistant (TA) in a programming language course and designed a system that allowed students to contribute their ideas. In the course, the students' grades were mainly determined by their programming assignments. 
Students submitted their codes, and the submissions are graded by scoring test cases. 
Because of programming assignments' characteristic, however, small mistakes (occurring a compile error) often resulted in a poor score (e.g., 0). Students who made such mistakes were often disappointed and requested opportunities to rescue their programs by fixing the minor errors. As I sympathized with the students and wanted to give them a chance, I designed a system named \Rescue~that automatically classifies whether a given code change is a minor revision that corrects only small mistakes. Student resubmitted their revised code and I accepted the code change if our system classifies it as a minor revision. 



A few weeks later, amazing things happened: the students started proposing creative ideas to improve the system and discussing its vulnerabilities. I would like to note that the system was an improvise one; improving or analyzing the system was not a main content of the course. The students, however, enjoyed using their talents or knowledge to enhance or analyze the system. For example, some students leveraged their programming language skills, which they had learned in the course, to enable the system capture minor revisions that corrected typos, which had not been possible before the idea. 
Other students used their knowledge of security to analyze the system's vulnerabilities and provided valuable ideas to remove them. 
I was excited by the proposed ideas and actively discussed and applied them to Rescue. 
I believe that such activities are the essence of education, enabling students to leverage their knowledge and talents to address their problems. 
The story of the system is now become a research paper consisting of various interesting ideas proposed by the students.
This experience has led me to believe that providing interesting content that can motivate the student is the most important part of teaching.

% If a content deeply motivates students, they will not just learn existing knowledge but leverage all their knowledge to give innovative ideas to address the problems.

%related to not only course but the students 


% The experience makes me believe that providing interesting unseen contents is the most import part of teaching. If the content enables the students to be deeply motivated by the contents, the course can provide an ideal education. 
% Students will not just learn existing knowledge to complete their assignments. They will try to leverage all their knowledge to give innovative idea for the unseen problem.





\myparagraph{Teaching Interests}
My research experience covers wide range of those topics in programming languages and software engineering. 
I have conducted various researches related to static program analysis, and I aspire to teach undergraduate and graduate courses related to my research topic to create interesting content for students.
Specifically, I want to teach {\it Programming Language}, {\it Software Engineering}, {\it Compiler}, {\it Theory of Computation}, and {\it Program Analysis} courses. 





\myparagraph{Advising}
I believe that each student is a unique research topic, and the most important challenge in advising is exploring talents and interests of the students.
During my Ph.D. program, I provided informal guidance to two students, and they successfully published their projects as research papers and obtained their master degrees. Currently, I am also advising two undergraduate students.
When I advised them, I observed each student possesses different talents and interests.
For example, a students has talent in implementation while another students has talent in theoretical analysis. 
Advising should find and maximize such talents and interests, which would lead to outstanding results.






% I strongly believe that every student is a unique and distinct research topic, and the most important challenge in advising is exploring each student's talent and interests. During my Ph.D. program, I provided informal guidance to two master's students who successfully published their research projects and obtained their degrees. Currently, I am advising two undergraduate students and have observed that each student possesses different talents. Some excel in implementation skills, while others have a talent for theoretical analysis. Advising should consider and maximize each student's individual talents and interests, which would lead to outstanding results.




% When I advising them, I rather tried to understand the student 
% instead of leading them.

%
%The students and advisor needs to understand what the student can do.
%I experienced understanding the background of the students determines their research topics. 



% The system I provided 
% Many of the submissions, however, include small mistakes that the small mistakes often make the . 


% As the programs are scored with scoring that result in poor scores.
% The students, however, often make small mistakes that result in their 
% The system I made automatically classifies minor mistakes in their submissions.




% I experienced and observed that well-designed contents, such as introducing a problem closely related to the courses or students, or assignments helped the students understand the essence of the course and inspired the students to give creative ideas. 
% The instructors and teaching assistants also inspired by the students' idea and improved the contents. 
% I also believe the academia is also greatly impacted by teaching because the students do research based on the taught knowledge. For example, I observed a biased knowledge led the academia dismiss a valuable technique. 











% Seminar??
\vspace{0.5cm}
%\begin{flushright}
%Sundar Iyer
%\end{flushright}

\end{small}

%\newpage
% Change font size?
% \tiny, \footnotesize, \small,\normalsize, \large, \Large, \LARGE, and \huge 
%\begin{small}
\begin{footnotesize}
\end{footnotesize}

\end{document}

