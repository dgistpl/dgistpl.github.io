\documentclass[11pt]{article}
\usepackage{setspace}

\usepackage{kotex}
\usepackage{url}
\usepackage{color}
\usepackage{breakurl}
\newcommand{\bburl}[1]{\textcolor{blue}{\url{#1}}}

\newcommand{\Rescue}{\textsc{Rescue}}

 \addtolength{\textwidth}{5cm}
 \addtolength{\textheight}{5cm}
 \addtolength{\hoffset}{-2.5cm}
  \addtolength{\voffset}{-3cm}
 \addtolength{\marginparwidth}{-2cm}

 \newcommand{\myparagraph}[1]{\medskip\noindent{\it \textbf{#1.}}}
\begin{document}
\title{자기소개서}

\author{전민석}

\bibliographystyle{plain}

\newcommand{\DisjunctiveModel}{\textsc{Disjunctive Model}}
\newcommand{\FeatureLanguage}{\textsc{Feature Language}}
\newcommand{\GDL}{\textsc{Graph Description Language}}
\newcommand{\PLXGL}{\textsc{PL4XGL}}


\newcommand{\AbstractRelativeWritePattern}{\textsc{Abstract Relative Write Pattern}}

\maketitle 
\begin{spacing}{1.125}
모든 문제는 적합한 언어를 디자인 및 사용하여 풀어야 한다.
%
문제와 알맞는 언어가 만나면, 문제의 본질을 꿰뚫어볼 수 있게 되고 그에 따라 문제를 제대로 해결하는 방법을 찾을 수 있게 되기 때문이다.
%
이는 컴퓨터학 분야에서도 마찬가지이다.
%



%
이 철학을 기반으로 컴퓨터 분야의 다양한 문제들의 해결책을 표현하기 위한 언어들을 디자인해왔고 언어 위에서 정답을 찾는 알리즘을 개발해왔다.
%
디자인한 언어들은 문제의 본질적인 해결책을 만들 수 있도록 해주었고, 이를 통해 혁신적인 연구 및 수업 결과물들을 만들어 내왔다.
%

\end{spacing}




%\bibliography{refs}


\end{document}


