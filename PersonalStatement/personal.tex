\documentclass[11pt]{article}
\usepackage{setspace}

\usepackage{kotex}
\usepackage{url}
\usepackage{color}
\usepackage{breakurl}
\newcommand{\bburl}[1]{\textcolor{blue}{\url{#1}}}

\newcommand{\Rescue}{\textsc{Rescue}}

 \addtolength{\textwidth}{5cm}
 \addtolength{\textheight}{5cm}
 \addtolength{\hoffset}{-2.5cm}
  \addtolength{\voffset}{-3cm}
 \addtolength{\marginparwidth}{-2cm}

 \newcommand{\myparagraph}[1]{\medskip\noindent{\it \textbf{#1.}}}
\begin{document}
\title{자기소개서}

\author{전민석}

\bibliographystyle{plain}

\newcommand{\DisjunctiveModel}{\textsc{Disjunctive Model}}
\newcommand{\FeatureLanguage}{\textsc{Feature Language}}
\newcommand{\GDL}{\textsc{Graph Description Language}}
\newcommand{\PLXGL}{\textsc{PL4XGL}}


\newcommand{\AbstractRelativeWritePattern}{\textsc{Abstract Relative Write Pattern}}

\maketitle 
\begin{spacing}{1.145}
모든 문제는 알맞은 언어를 디자인 및 사용하여 풀어야 한다.
%
문제와 알맞는 언어가 만나면, 문제의 본질을 꿰뚫어볼 수 있게 되고 그에 따라 문제를 제대로 해결하는 방법을 찾을 수 있게 되기 때문이다.
%
이는 컴퓨터학 분야에서도 마찬가지이다.
%
컴퓨터학 각 분야의 문제들의 본질을 꿰뚫어보고, 그에 따른 해결책을 찾기 위해 적합한 언어를 디자인하고 사용하는 것이 중요하다.
%
이 철학을 기반으로 컴퓨터학 분야의 다양한 문제들의 해결책을 표현하기 위한 언어들을 디자인해왔고, 언어 위에서 정답을 찾는 알고리즘을 개발해왔다.
%
디자인한 언어들은 문제의 본질적인 해결책을 만들 수 있도록 해주었고, 이를 통해 혁신적인 연구 및 수업 결과물들을 만들어 내었다.
%
컴퓨터학 분야에서 문제를 해결하기 위한 언어를 디자인한다는 것은 각 분야의 특화된 프로그래밍 언어(domain-specific programming language)를 디자인하는 것을 의미한다.
%
디자인 한 도메인 특화 프로그래밍 언어들은 문제의 정답을 프로그램으로써 표현할 수 있도록 디자인 되어야 하며, 이를 위해 문제의 본질적인 특성을 반영하여야 한다.
%
이러한 언어들은 각 문제의 정답(해결책)을 표현하기 위한 문법과 의미론을 제공하며, 디자인 한 언어 위에서 문제의 해결책을 찾아내는 알고리즘을 개발할 수 있게 해준다.



설명 가능한 그래프 기계학습, 소프트웨어 테스팅, 소프트웨어 정적 분석, 프로그래밍 교육등 다양한 컴퓨터학 분야의 문제들의 본질적인 해결책을 제시해기 위해 도메인 특화 프로그래밍 언어들을 디자인하고 알고리즘을 개발해왔다.
%
설명 가능한 그래프 기계학습을 위해 다지인 한 특화 프로그래밍 언어는 본질적으로 설명 가능한 그래프 기계학습 모델을 표현할 수 있도록 디자인 되었으며, 이를 통해 학습한 그래프 기계학습 모델은 내놓은 모든 결과에 대해 올바른 설명을 제공할 수 있게 된다.
%
정적 분석을 위한 특화 프로그래밍 언어는, 분석을 빠르고 정확하게 해주는 정적 분석 전략을 표현할 수 있도록 디자인 되었으며, 소프트웨어 테스팅 문제를 해결하기 위한 특화 프로그래밍 언어는 효과적인 테스트 케이스를 쉽게 생성할 수 있도록 디자인 되었다.
%
학계에서는 디자인한 특화 프로그래밍 언어의 우수성을 인정 받아, 소위 최우수 국제 학술대회들(PLDI 2024, POPL 2022, OOPSLA 2020, OOPSLA 2018, OOPSLA 1017)에서 논문들을 발표해왔다.
%
제안한 언어 위에서 후속 연구들이 활발히 진행되고 있으며, 산업계에서도 제안한 언어를 사용하여 문제를 해결하고 있다.
%
프로그래밍 수업에서도 또한 수업에서 발생하는 문제들을 해결하기 위해 디자인한 특화 언어로 학생들이 프로그래밍을 배우면서 발생하는 문제들을 해결하고 있다.





\section{설명 가능한 그래프 기계학습을 위한 특화 프로그래밍 언어}
%
프로그램 분석, 이상 거래 탐지, 프로그램 수정, 약물 발견 등 다양한 현실 세계의 문제들은 그래프 문제로 표현될 수 있다.
%
이러한 문제들은 그래프 기계학습 문제로 모델링될 수 있으며, 현재 인공신경망 기반의 그래프 기계학습 모델(Graph Neural Networks)이 많은 그래프 문제들을 해결하고 있다~\cite{wu2019comprehensive}.
%
그러나, 이러한 인공 신경망 기반 모델들은 내놓은 결과에 대한 설명을 제공하지 않는 단점이 있다.
%
이상 거래 탐지 및 신약 개발과 같은 사람의 안전과 관련된 중요한 문제들에서는 모델의 결과에 대한 설명이 필수적이기 때문에, 인공신경망 기반 그래프 기계학습 방법은 적합하지 않다.
%
그래프 기계학습 모델의 결과에 대한 설명을 제공하는 방법들이 다양하게 제안되어 왔지만, 인공신경망 설명의 본질적 어려움 때문에 제안된 방법들 또한 모델의 결과에 대한 실제 이유를 반영하는 제대로 된 설명을 제공하지 못한다~\cite{yuan20survey}.







\clearpage
Our approach can also be generalized to develop inherently explainable machine learning methods. 
%
Many significant real-world problems can be modeled as graph machine learning problems, including fraud detection, program repair, and drug discovery.
%
In such decision-critical applications, it is important to understand why the models made each prediction.
%
However, the dominant neural-network based graph machine learning models (i.e., graph neural networks) are black-box models that do not explain why they made each prediction.
%
Various GNN explanation techniques have been proposed, but they fall short in providing satisfactory explanations because explaining the black-box models is fundamentally challenging.
%
To address this problem, I developed a new inherently explainable machine learning method {\PLXGL}.
%
The key idea of \PLXGL~is to generate inherently explainable graph machine learning models by designing a domain-specific language, called \GDL, and synthesis algorithms that learn models from training graph data.
%
The experimental results show that our DSL-based approach successfully generates inherently explainable graph machine learning models that show competitive accuracy compared to the existing dominant neural network-based models but provide significantly better explanations for the predictions.
%
Based on this work, I will lead DSL-based explainable machine learning.



\section{고성능 정적 분석을 위한 특화 프로그래밍 언어}




\section{고성능 소프트웨어 테스팅을 위한 특화 프로그래밍 언어}



\section{프로그래밍 수업에서 발생하는 문제 해결을 위한 특화 프로그래밍 언어}



\section{최종 목표}




\end{spacing}




\bibliography{refs}


\end{document}


