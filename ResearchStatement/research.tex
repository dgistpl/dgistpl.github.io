\documentclass[11pt]{article}


\usepackage{kotex}
\usepackage{url}
\usepackage{color}
\usepackage{breakurl}
\newcommand{\bburl}[1]{\textcolor{blue}{\url{#1}}}

\newcommand{\Rescue}{\textsc{Rescue}}

 \addtolength{\textwidth}{5cm}
 \addtolength{\textheight}{5cm}
 \addtolength{\hoffset}{-2.5cm}
  \addtolength{\voffset}{-3cm}
 \addtolength{\marginparwidth}{-2cm}

 \newcommand{\myparagraph}[1]{\medskip\noindent{\it \textbf{#1.}}}
\begin{document}
\title{Research Statement}
%\title{연구계획서}

\author{Minseok Jeon (전민석)}

\bibliographystyle{plain}

\newcommand{\DisjunctiveModel}{\textsc{Disjunctive Model}}
\newcommand{\FeatureLanguage}{\textsc{Feature Language}}
\newcommand{\GDL}{\textsc{Graph Description Language}}
\newcommand{\PLXGL}{\textsc{PL4XGL}}


\newcommand{\AbstractRelativeWritePattern}{\textsc{Abstract Relative Write Pattern}}

\maketitle 


To effectively address a problem, a domain-specific language designed for that problem is essential. 
%
Using a suitable language allows us to grasp the essence of the problem and find the appropriate solution. 
%
This principle is also applicable in computer science.
%
Each problem in computer science has its own characteristics, and a domain-specific language tailored to these characteristics is necessary to solve the problem effectively.




My research aims to design domain-specific programming languages (DSLs) and program synthesis algorithms to address challenges in various computer science domains, including programming languages, software engineering, and machine learning.
%
Given a problem, I design a DSL that can describe the solution to the target problem.
%
The domain-specific language is carefully designed to capture the essence of the problem.
%
Then, I develop a synthesis algorithm that automatically and effectively searches for solutions within the DSL.
%
In the programming language domain, for example, I have developed DSLs and program synthesis algorithms for generating powerful analysis heuristics in pointer analysis, a key ingredient in compiler optimization.
%
In software engineering, I have designed DSLs for generating effective test cases for firmware (system software) endurance tests. 
%
In machine learning, I used my DSL-based approach to develop an inherently explainable graph machine learning method.
%
I have demonstrated that the DSL-based approach produces outstanding results in various domains.
%
Currently, I am applying the DSL-based approach to fault localization or trying and working on improving the designed DSLs and synthesis algorithms.



As a long-term goal, I will develop a general framework that can automatically generate DSLs for any given problem. 
%
Currently, I design DSLs manually for each problem, which is a very challenging and time-consuming task. 
%
This is primarily because we do not know the solutions to the problem when we begin designing a DSL.
%
Currently, I repeatedly guess and refine a DSL, then check its effectiveness, which is time-consuming.
%
As a long-term goal, I aim to develop a framework that automates this language designing process.
%
Based on this framework, I will establish the DSL-based approach as a dominant method for addressing most problems in computer science.


\section{DSLs \& Synthesis Algorithms for Effective Static Analysis}
%
The goal of pointer analysis is to approximate the set of memory locations that each variable may point to during the program executions.
%
Pointer analysis is a key ingredient in compiler optimization, and it is also widely used in many other software engineering techniques, including bug detection, security analysis, and program repair.
%
The success of pointer analyzers heavily depends on the quality of their underlying analysis heuristics.
%
Without high-quality analysis heuristics, the analysis becomes imprecise or expensive.
%
Before my research, those heuristics had been manually designed by domain experts. 
%
However, manually designing analysis heuristics requires time-consuming and laborious tasks.
%
Even worse, the manually crafted heuristics have shown suboptimal performance.
%
To address this problem, I have designed domain-specific languages that are expressive enough to describe high-quality analysis heuristics and program synthesis algorithms that effectively search  high-quality analysis heuristics in the DSLs.



\myparagraph{Disjunctive Model $\&$ Synthesis Algorithms~\cite{JeJeChOh17,Jeon2019,IST2021}}
%
Context sensitivity is one of the most impactful factors in pointer analysis that improves the precision by distinguishing variables and objects in different calling contexts.
%
Applying context sensitivity to all methods, however, is too expensive in practice.
%
Selective context sensitivity aims to address this problem by selectively applying context sensitivity to methods.
%
That is, selective context sensitivity applies context sensitivity to only a subset of methods that are likely to benefit from context sensitivity, while applying context insensitivity to the remaining methods.
%
To effectively apply selective context sensitivity, however, qualified analysis heuristics are required to accurately determine whether each method should be analyzed with context sensitivity.




To address this problem, I designed a domain-specific language, called \DisjunctiveModel, that describes various selective context sensitivity heuristics.
%
A heuristic in the DSL is a sequence of disjunctive normal form (DNF) formulas consist of user-provided features describing methods that will be analyzed context sensitively.
%
Simultaneously, I designed program synthesis algorithms that automatically generates qualified selective context sensitivity heuristics from the DSL.
%
The experimental results show that the DSL-based approach automatically generates outstanding analysis heuristics that outperform the existing state-of-the-art manually crafted heuristics.
%



%\myparagraph{Synthesis Algorithm for Generating Context Tunneling Heuristics~\cite{JeJeOh18}}
\myparagraph{Context Tunneling~\cite{JeJeOh18}}
%
In static analysis, context abstraction is essential as it is impractical to keep all the concrete contexts, and the dominant context abstraction method is $k$-limited context sensitivity, which keeps only the last $k$ context elements.
%
However, I found the last-$k$ context abstraction has a severe limitation. 
%
This approach removes key context elements if they fall outside the last $k$ context elements, significantly degrading the precision of the analysis.




To address this problem, I present context tunneling which enables the $k$-limited context sensitivity to keep the most important $k$-context elements instead of the last-$k$ context elements.
%
To apply context tunneling, however, context tunneling heuristics are required to check whether each context element is important.
%
To this end, I adapted the \DisjunctiveModel~and the synthesis algorithm~\cite{JeJeChOh17} to learn high-quality context tunneling heuristics.
%
The experimental results show that the produced context tunneling heuristics significantly improved the effectiveness of the $k$-limited context-sensitive analyses.



% \myparagraph{Synthesis Algorithm for Transforming a Given Object Sensitivity into a More Precise Call-Site Sensitivity~\cite{JeOh22}}

\myparagraph{Obj2CFA~\cite{JeOh22}}
%
In OOP (Object-Oriented Programming) program analysis, object sensitivity has been established as the dominant context flavor.
%
The superiority of object sensitivity over other context flavors has been reinforced by extensive research.
%
On the other hand, call-site sensitivity has been consistently dismissed because it has shown poor performance in both precision and scalability.
%
However, I found that this superiority of object sensitivity over call-site sensitivity does not hold when context tunneling~\cite{JeJeOh18} is included.


In this work, I challenged the commonly accepted knowledge by developing a synthesis algorithm, called \textsc{Obj2CFA}, which transforms a given object sensitivity into a more precise call-site sensitivity.
%
The synthesis algorithm takes a tunneling heuristic of an object sensitivity and produces a corresponding tunneling heuristic for call-site sensitivity.
%
The experimental results show that the transformed call-site sensitivity is significantly more precise and scalable than the given object sensitivity.




%\myparagraph{DSL $\&$ Synthesis Algorithm for Automatically Generating Features~\cite{Jeon20}}
\myparagraph{Graphick~\cite{Jeon20}}
%
This work addresses a key burden of the previous DSL-based approach. 
%
\DisjunctiveModel~combines user-provided features to describe analysis heuristics; the success of \DisjunctiveModel~heavily depends on the quality of user-provided features.
%
However, designing high-quality features is a nontrivial and laborious task for the users.
%
To remove this burden, I designed a framework \textsc{Graphick} that automatically generates high-quality features.
%
To this end, I designed a domain-specific language, called \FeatureLanguage, that describes various features.
%
Subsequently, I developed a synthesis algorithm that automatically generates qualified features from the DSL.
%
The experimental results demonstrate that our approach successfully generates high-quality features, enabling the learning of effective analysis heuristics without user-provided features.




\section{DSL \& Synthesis Algorithms for Software Testing}
%
In system software testing like write endurance tests, designing test cases has also been done manually by domain experts; it faces the same challenges.
%.
It requires laborious efforts, and the manually crafted test cases often yield suboptimal results.
%
To address this problem, I also applied my DSL-based approach to automatically generate effective test cases in system software testing.




%\myparagraph{DSL $\&$ Synthesis Algorithm for Generating Effective Test Cases in Endurance Test for Flash-based Storage Devices~\cite{ARES23}}
\myparagraph{ARES~\cite{ARES23}}
%
Flash-based storage devices are widely used in practice, including mobile devices, automotives, and PC. 
%
However, flash-based storage devices have a finite limit in processing data write requests.
%
Therefore, it is important for manufacturers to rigorously test and accurately provide the maximum data write capacity that their products can reliably endure.
%
To automate the generation of effective test cases for the endurance test, I designed a domain-specific language named \AbstractRelativeWritePattern, which effectively reduces the search space of test cases.
%
Then, I developed a synthesis algorithm that automatically generates qualified test cases using the DSL.
%
The experimental results show that the DSL-based approach successfully generates qualified test cases that outperform the existing manually designed test cases used in practice.



\section{DSL \& Synthesis Algorithms for Explainable Machine Learning}
%
Our approach can also be generalized to develop inherently explainable machine learning methods. 
%

\myparagraph{PL4XGL}
Many significant real-world problems can be modeled as graph machine learning problems, including fraud detection, program repair, and drug discovery.
%
In such decision-critical applications, it is important to understand why the models made each prediction.
%
However, the dominant neural-network based graph machine learning models (i.e., graph neural networks) are black-box models that do not explain why they made each prediction.
%
Various GNN explanation techniques have been proposed, but they fall short in providing satisfactory explanations because explaining the black-box models is fundamentally challenging.
%
To address this problem, I developed a new inherently explainable machine learning method {\PLXGL}.
%
The key idea of \PLXGL~is to generate inherently explainable graph machine learning models by designing a domain-specific language, called \GDL, and synthesis algorithms that learn models from training graph data.
%
The experimental results show that our DSL-based approach successfully generates inherently explainable graph machine learning models that show competitive accuracy compared to the existing dominant neural network-based models but provide significantly better explanations for the predictions.
%
Based on this work, I will lead DSL-based explainable machine learning.



\section{Future Research Plan}
I believe that my DSL-based approaches are promising and have a lot of opportunities for improvement, with the potential to extend their applicability to various other domains.
%
In my future work, I will address the existing challenges and limitations in the current methods, and I will also expand the use of the DSL-based approach to other domains.


\subsection{Improving the DSL-based Approaches in Static Analysis}

The current DSL-based techniques for pointer analysis have several limitations, and I plan to address these limitations.

\myparagraph{Generating Combinations of Analysis Heuristics}
Practical static analyzers use combinations of various analysis heuristics.
%
This necessitates heuristic design processes to consider the interactions among these heuristics.
%
However, my previous approaches lack consideration of the combinations.
%
They have focused on generating a single analysis heuristic without considering interactions with other heuristics.
%
Recently, I observed that blindly combining the analysis heuristics results in suboptimal performance.
%
To address this problem, I will improve my DSL-based approach to generate combinations of analysis heuristics.
%
To achieve this, I will identify and clarify the properties and relationships between different analysis heuristics.
% 
Then, I will design DSLs and synthesis algorithms based on these properties and relationships.




\myparagraph{Understanding the Principles Behind the Generated Heuristics}
Though effective, the analysis heuristics produced from the DSL-based approaches are currently considered as black-box heuristics that do not explain why they are effective in pointer analysis.
%
% Unlike neural network-based approaches, our DSL-based approaches are interpretable as they are described using symbols (e.g., DNF formulas); Understanding the symbols in the learned heuristics can lead to an understanding of the principles behind the learned heuristics.
%
Understanding the principles behind the learned heuristics, however, is important as it provides key insights when designing analysis heuristics.
% I would like to note that understanding the principles behind the learned heuristics is valuable as it will provide key insights when designing analysis heuristics.
%
Currently, the learned analysis heuristics consist of complex DNF formulas, and this complexity poses a significant challenge in understanding the essence of the learned heuristics.
%
To understand the principle behind the learned heuristics, I plan to design a framework that transform a given heuristic into a simplified one while maintaining its effectiveness.
%
The simplified heuristics will help the understanding of the learned principles.
%
Additionally, this simplification process may lead to the generalization of the learned heuristics, potentially enhancing their performance.



\subsection{Generalizing the DSL-based Approaches to Other Domains}
I am currently working on extending my DSL-based approach to other domains.
%
Specifically, I am trying to apply my DSL-based approach to fault localization and explainable graph machine learning.

\myparagraph{Generalization to Fault Localization}
The goal of fault localization is to identify the locations of faults in programs that produce errors.
%
Currently, \textit{Spectrum-Based Fault Localization} (SBFL), which localizes buggy lines using test cases execution information, is the dominant fault localization technique because of its outstanding efficiency.
%
SBFL, however, has a severe limitation in accuracy; it often fails to accurately identify the locations of faults.
%
To address this problem, Idevelop a DSL-based approach to improve the accuracy of SBFL.


Our approach is based on an observation that similar faults have redundantly appeared in the same projects; faults in a previous version of a project are likely to occur again in subsequent versions.
%
Based on this observation, I designed a domain-specific language that describes various fault patterns.
%
Using the DSL, our synthesis algorithm learns fault patterns from the previous versions of a project.
%
These learned fault patterns are then applied to enhance the accuracy of SBFL in the next version of the project.
%
The preliminary results show that our project-aware fault localication technique successfully improves the accuracy of SBFL.
% When a new version of a project is released, our approach automatically generates a spectrum-based fault localization technique tailored specifically for the project by utilizing the learend patterns.






\bibliography{refs}


\end{document}


